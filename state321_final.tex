% Options for packages loaded elsewhere
\PassOptionsToPackage{unicode}{hyperref}
\PassOptionsToPackage{hyphens}{url}
%
\documentclass[
]{article}
\usepackage{amsmath,amssymb}
\usepackage{lmodern}
\usepackage{iftex}
\ifPDFTeX
  \usepackage[T1]{fontenc}
  \usepackage[utf8]{inputenc}
  \usepackage{textcomp} % provide euro and other symbols
\else % if luatex or xetex
  \usepackage{unicode-math}
  \defaultfontfeatures{Scale=MatchLowercase}
  \defaultfontfeatures[\rmfamily]{Ligatures=TeX,Scale=1}
\fi
% Use upquote if available, for straight quotes in verbatim environments
\IfFileExists{upquote.sty}{\usepackage{upquote}}{}
\IfFileExists{microtype.sty}{% use microtype if available
  \usepackage[]{microtype}
  \UseMicrotypeSet[protrusion]{basicmath} % disable protrusion for tt fonts
}{}
\makeatletter
\@ifundefined{KOMAClassName}{% if non-KOMA class
  \IfFileExists{parskip.sty}{%
    \usepackage{parskip}
  }{% else
    \setlength{\parindent}{0pt}
    \setlength{\parskip}{6pt plus 2pt minus 1pt}}
}{% if KOMA class
  \KOMAoptions{parskip=half}}
\makeatother
\usepackage{xcolor}
\usepackage[margin=1in]{geometry}
\usepackage{color}
\usepackage{fancyvrb}
\newcommand{\VerbBar}{|}
\newcommand{\VERB}{\Verb[commandchars=\\\{\}]}
\DefineVerbatimEnvironment{Highlighting}{Verbatim}{commandchars=\\\{\}}
% Add ',fontsize=\small' for more characters per line
\usepackage{framed}
\definecolor{shadecolor}{RGB}{248,248,248}
\newenvironment{Shaded}{\begin{snugshade}}{\end{snugshade}}
\newcommand{\AlertTok}[1]{\textcolor[rgb]{0.94,0.16,0.16}{#1}}
\newcommand{\AnnotationTok}[1]{\textcolor[rgb]{0.56,0.35,0.01}{\textbf{\textit{#1}}}}
\newcommand{\AttributeTok}[1]{\textcolor[rgb]{0.77,0.63,0.00}{#1}}
\newcommand{\BaseNTok}[1]{\textcolor[rgb]{0.00,0.00,0.81}{#1}}
\newcommand{\BuiltInTok}[1]{#1}
\newcommand{\CharTok}[1]{\textcolor[rgb]{0.31,0.60,0.02}{#1}}
\newcommand{\CommentTok}[1]{\textcolor[rgb]{0.56,0.35,0.01}{\textit{#1}}}
\newcommand{\CommentVarTok}[1]{\textcolor[rgb]{0.56,0.35,0.01}{\textbf{\textit{#1}}}}
\newcommand{\ConstantTok}[1]{\textcolor[rgb]{0.00,0.00,0.00}{#1}}
\newcommand{\ControlFlowTok}[1]{\textcolor[rgb]{0.13,0.29,0.53}{\textbf{#1}}}
\newcommand{\DataTypeTok}[1]{\textcolor[rgb]{0.13,0.29,0.53}{#1}}
\newcommand{\DecValTok}[1]{\textcolor[rgb]{0.00,0.00,0.81}{#1}}
\newcommand{\DocumentationTok}[1]{\textcolor[rgb]{0.56,0.35,0.01}{\textbf{\textit{#1}}}}
\newcommand{\ErrorTok}[1]{\textcolor[rgb]{0.64,0.00,0.00}{\textbf{#1}}}
\newcommand{\ExtensionTok}[1]{#1}
\newcommand{\FloatTok}[1]{\textcolor[rgb]{0.00,0.00,0.81}{#1}}
\newcommand{\FunctionTok}[1]{\textcolor[rgb]{0.00,0.00,0.00}{#1}}
\newcommand{\ImportTok}[1]{#1}
\newcommand{\InformationTok}[1]{\textcolor[rgb]{0.56,0.35,0.01}{\textbf{\textit{#1}}}}
\newcommand{\KeywordTok}[1]{\textcolor[rgb]{0.13,0.29,0.53}{\textbf{#1}}}
\newcommand{\NormalTok}[1]{#1}
\newcommand{\OperatorTok}[1]{\textcolor[rgb]{0.81,0.36,0.00}{\textbf{#1}}}
\newcommand{\OtherTok}[1]{\textcolor[rgb]{0.56,0.35,0.01}{#1}}
\newcommand{\PreprocessorTok}[1]{\textcolor[rgb]{0.56,0.35,0.01}{\textit{#1}}}
\newcommand{\RegionMarkerTok}[1]{#1}
\newcommand{\SpecialCharTok}[1]{\textcolor[rgb]{0.00,0.00,0.00}{#1}}
\newcommand{\SpecialStringTok}[1]{\textcolor[rgb]{0.31,0.60,0.02}{#1}}
\newcommand{\StringTok}[1]{\textcolor[rgb]{0.31,0.60,0.02}{#1}}
\newcommand{\VariableTok}[1]{\textcolor[rgb]{0.00,0.00,0.00}{#1}}
\newcommand{\VerbatimStringTok}[1]{\textcolor[rgb]{0.31,0.60,0.02}{#1}}
\newcommand{\WarningTok}[1]{\textcolor[rgb]{0.56,0.35,0.01}{\textbf{\textit{#1}}}}
\usepackage{graphicx}
\makeatletter
\def\maxwidth{\ifdim\Gin@nat@width>\linewidth\linewidth\else\Gin@nat@width\fi}
\def\maxheight{\ifdim\Gin@nat@height>\textheight\textheight\else\Gin@nat@height\fi}
\makeatother
% Scale images if necessary, so that they will not overflow the page
% margins by default, and it is still possible to overwrite the defaults
% using explicit options in \includegraphics[width, height, ...]{}
\setkeys{Gin}{width=\maxwidth,height=\maxheight,keepaspectratio}
% Set default figure placement to htbp
\makeatletter
\def\fps@figure{htbp}
\makeatother
\setlength{\emergencystretch}{3em} % prevent overfull lines
\providecommand{\tightlist}{%
  \setlength{\itemsep}{0pt}\setlength{\parskip}{0pt}}
\setcounter{secnumdepth}{-\maxdimen} % remove section numbering
\ifLuaTeX
  \usepackage{selnolig}  % disable illegal ligatures
\fi
\IfFileExists{bookmark.sty}{\usepackage{bookmark}}{\usepackage{hyperref}}
\IfFileExists{xurl.sty}{\usepackage{xurl}}{} % add URL line breaks if available
\urlstyle{same} % disable monospaced font for URLs
\hypersetup{
  pdftitle={STAT321\_FinalReport},
  hidelinks,
  pdfcreator={LaTeX via pandoc}}

\title{STAT321\_FinalReport}
\author{}
\date{\vspace{-2.5em}2023-06-08}

\begin{document}
\maketitle

\hypertarget{final-report-investigating-the-relationship-between-land-use-diversity-and-pedestrian-activity-in-seattle}{%
\section{FINAL Report: Investigating The Relationship Between Land Use
Diversity and Pedestrian Activity in
Seattle}\label{final-report-investigating-the-relationship-between-land-use-diversity-and-pedestrian-activity-in-seattle}}

\hypertarget{authors-finch-brown-and-jp-lopez}{%
\subsubsection{Authors: Finch Brown and JP
Lopez}\label{authors-finch-brown-and-jp-lopez}}

\hypertarget{introduction-and-research-design}{%
\subsection{Introduction and Research
Design}\label{introduction-and-research-design}}

The primary research question guiding this investigation is: Does land
use diversity influence the level of pedestrian activity in urban
spaces? This study's focus stems from the conjecture that diversity in
land use could be a significant factor influencing the movement of
pedestrians. From a theoretical perspective, diverse land uses offer
more opportunities for activity and thus draw more pedestrian traffic.
Therefore, our hypothesis for this study is as follows:

H0: Land use diversity has no significant impact on pedestrian activity.
(β\_diversity = 0) H1: Land use diversity positively influences
pedestrian activity. (β\_diversity \textgreater{} 0)

The importance of this research lies in its potential implications for
urban development and planning, ultimately encouraging active and
sustainable urban environments.

The research design for this investigation utilizes observational data
from real-world observations in Seattle. We have categorized each
location according to the classifications used by the Gehl Institute's
Public Life Data Protocol, distinguishing them into single-use or
mixed-use areas. Our measurement of pedestrian activity is based on
total counts of moving pedestrians recorded in each entry of the
dataset.

If the data analysis reveals a significant positive relationship between
land use diversity and pedestrian activity, it will provide empirical
support for our hypothesis, underscoring the contribution of diverse
land use to increased pedestrian movement.

\hypertarget{data-cleaning}{%
\subsection{Data Cleaning}\label{data-cleaning}}

\hypertarget{a-data-section-describing-the-source-of-the-data-measurement-and-uses-plots-to-summarize-the-dependent-variable}{%
\subsubsection{A data section describing the source of the data,
measurement, and uses plots to summarize the dependent
variable}\label{a-data-section-describing-the-source-of-the-data-measurement-and-uses-plots-to-summarize-the-dependent-variable}}

\hypertarget{exploratory-data-analysis-analysis-let-the-data-speak-by-constructing-relevant-and-polished-plots-or-visuals-that-reveal-important-patterns-and-relationships-related-to-the-research-question-and-hypotheses.-provide-a-narrative-explanation-of-these-patterns-and-their-potential-influence-on-the-model-specification.}{%
\subsubsection{Exploratory Data Analysis: analysis: Let the data speak
by constructing relevant and polished plots or visuals that reveal
important patterns and relationships related to the research question
and hypotheses. Provide a narrative explanation of these patterns and
their potential influence on the model
specification.}\label{exploratory-data-analysis-analysis-let-the-data-speak-by-constructing-relevant-and-polished-plots-or-visuals-that-reveal-important-patterns-and-relationships-related-to-the-research-question-and-hypotheses.-provide-a-narrative-explanation-of-these-patterns-and-their-potential-influence-on-the-model-specification.}}

\begin{Shaded}
\begin{Highlighting}[]
\DocumentationTok{\#\# Prerequisite}
\DocumentationTok{\#\#\# Load Libraries}
\FunctionTok{library}\NormalTok{(tidyverse)}
\end{Highlighting}
\end{Shaded}

\begin{verbatim}
## -- Attaching core tidyverse packages ------------------------ tidyverse 2.0.0 --
## v dplyr     1.1.1     v readr     2.1.4
## v forcats   1.0.0     v stringr   1.5.0
## v ggplot2   3.4.2     v tibble    3.2.1
## v lubridate 1.9.2     v tidyr     1.3.0
## v purrr     1.0.1     
## -- Conflicts ------------------------------------------ tidyverse_conflicts() --
## x dplyr::filter() masks stats::filter()
## x dplyr::lag()    masks stats::lag()
## i Use the conflicted package (<http://conflicted.r-lib.org/>) to force all conflicts to become errors
\end{verbatim}

\begin{Shaded}
\begin{Highlighting}[]
\FunctionTok{library}\NormalTok{(dplyr)}
\FunctionTok{library}\NormalTok{(ggplot2)}

\DocumentationTok{\#\#\# loading datasets}
\NormalTok{people\_moving\_df }\OtherTok{\textless{}{-}} \FunctionTok{read.csv}\NormalTok{(}\StringTok{"Public\_Life.csv"}\NormalTok{)}
\FunctionTok{View}\NormalTok{(people\_moving\_df)}

\NormalTok{locations\_df }\OtherTok{\textless{}{-}} \FunctionTok{read.csv}\NormalTok{(}\StringTok{"locations.csv"}\NormalTok{)}
\FunctionTok{View}\NormalTok{(locations\_df)}
\end{Highlighting}
\end{Shaded}

The datasets used in this report were sourced from the Public Life Data
repository by Seattle's Department of Transportation (link:
\url{https://data.seattle.gov/Transportation/Public-Life-Data-People-Moving/7rx6-5pgd}).
The primary dataset, `Public Life Data: People Moving,' comprises
pedestrian counts from different Seattle locations. The second dataset,
`Locations,' contains various land use types (e.g., Residential, Mixed,
Commercial, etc.).

To illustrate the dependent variable, i.e., pedestrian movement, a
boxplot is presented, demonstrating pedestrian counts in single-use and
mixed-use spaces.

\begin{Shaded}
\begin{Highlighting}[]
\DocumentationTok{\#\#\# isolate relevant columns}
\FunctionTok{unique}\NormalTok{(locations\_df}\SpecialCharTok{$}\NormalTok{location\_character)}
\end{Highlighting}
\end{Shaded}

\begin{verbatim}
## [1] "Commercial"      "Mixed"           "Infrastructural" "CBD"            
## [5] "Residential"     "Medical"         "Recreational"    "Civic"
\end{verbatim}

\begin{Shaded}
\begin{Highlighting}[]
\NormalTok{small\_loc\_df }\OtherTok{\textless{}{-}} \FunctionTok{data.frame}\NormalTok{(locations\_df}\SpecialCharTok{$}\NormalTok{location\_id, locations\_df}\SpecialCharTok{$}\NormalTok{location\_line\_typology\_vehicular)}
\NormalTok{small\_move\_df }\OtherTok{\textless{}{-}} \FunctionTok{data.frame}\NormalTok{(people\_moving\_df}\SpecialCharTok{$}\NormalTok{unique\_moving\_id, people\_moving\_df}\SpecialCharTok{$}\NormalTok{moving\_row\_total, people\_moving\_df}\SpecialCharTok{$}\NormalTok{moving\_time\_start, people\_moving\_df}\SpecialCharTok{$}\NormalTok{moving\_time\_end, people\_moving\_df}\SpecialCharTok{$}\NormalTok{location\_id)}

\DocumentationTok{\#\#\# rename columns}
\FunctionTok{colnames}\NormalTok{(small\_loc\_df) }\OtherTok{\textless{}{-}} \FunctionTok{c}\NormalTok{(}\StringTok{"location"}\NormalTok{, }\StringTok{"typology"}\NormalTok{)}
\FunctionTok{colnames}\NormalTok{(small\_move\_df) }\OtherTok{\textless{}{-}} \FunctionTok{c}\NormalTok{(}\StringTok{"id"}\NormalTok{, }\StringTok{"total\_count"}\NormalTok{, }\StringTok{"start"}\NormalTok{, }\StringTok{"end"}\NormalTok{, }\StringTok{"location"}\NormalTok{)}

\DocumentationTok{\#\#\# join the dataframes}
\NormalTok{new\_df }\OtherTok{\textless{}{-}} \FunctionTok{left\_join}\NormalTok{(small\_loc\_df, small\_move\_df, }\AttributeTok{by =} \FunctionTok{join\_by}\NormalTok{(location), }\AttributeTok{multiple =} \StringTok{"all"}\NormalTok{)}

\DocumentationTok{\#\#\# creating column to categorize neighborhood types}
\FunctionTok{unique}\NormalTok{(new\_df}\SpecialCharTok{$}\NormalTok{typology)}
\end{Highlighting}
\end{Shaded}

\begin{verbatim}
##  [1] "Neighborhood Corridor"             "Urban Village Neighborhood"       
##  [3] "Urban Village Neighborhood Access" "Downtown Neighborhood"            
##  [5] "Downtown Neighborhood Access"      "Downtown"                         
##  [7] "Urban Center Connector"            "Urban Village Main"               
##  [9] "Neighborhood Yield Street"         "Industrial Access"
\end{verbatim}

\begin{Shaded}
\begin{Highlighting}[]
\DocumentationTok{\#\#\# classify each location as either single{-}use or mixed{-}use}
\NormalTok{new\_df}\SpecialCharTok{$}\NormalTok{classification }\OtherTok{\textless{}{-}} \FunctionTok{ifelse}\NormalTok{(new\_df}\SpecialCharTok{$}\NormalTok{typology }\SpecialCharTok{\%in\%} \FunctionTok{c}\NormalTok{(}\StringTok{"Neighborhood Corridor"}\NormalTok{, }\StringTok{"Urban Village Neighborhood Access"}\NormalTok{, }\StringTok{"Urban Village Neighborhood"}\NormalTok{, }\StringTok{"Downtown Neighborhood"}\NormalTok{, }\StringTok{"Downtown"}\NormalTok{, }\StringTok{"Downtown Neighborhood Access"}\NormalTok{, }\StringTok{"Urban Center Connector"}\NormalTok{, }\StringTok{"Urban Village Main"}\NormalTok{), }\DecValTok{1}\NormalTok{, }\DecValTok{0}\NormalTok{)}

\DocumentationTok{\#\#\# add a column for count per unit of time}
\NormalTok{new\_df}\SpecialCharTok{$}\NormalTok{start }\OtherTok{\textless{}{-}} \FunctionTok{as.POSIXlt}\NormalTok{(new\_df}\SpecialCharTok{$}\NormalTok{start, }\AttributeTok{format=}\StringTok{"\%d/\%m/\%Y \%I:\%M:\%S \%p"}\NormalTok{, }\AttributeTok{tz =} \StringTok{"UTC"}\NormalTok{, }\AttributeTok{na.rm =}\NormalTok{ T)}
\NormalTok{new\_df}\SpecialCharTok{$}\NormalTok{end }\OtherTok{\textless{}{-}} \FunctionTok{as.POSIXlt}\NormalTok{(new\_df}\SpecialCharTok{$}\NormalTok{end, }\AttributeTok{format=}\StringTok{"\%d/\%m/\%Y \%I:\%M:\%S \%p"}\NormalTok{, }\AttributeTok{tz =} \StringTok{"UTC"}\NormalTok{, }\AttributeTok{na.rm =}\NormalTok{ T)}
\NormalTok{new\_df}\SpecialCharTok{$}\NormalTok{duration }\OtherTok{\textless{}{-}} \FunctionTok{as.numeric}\NormalTok{((new\_df}\SpecialCharTok{$}\NormalTok{end }\SpecialCharTok{{-}}\NormalTok{ new\_df}\SpecialCharTok{$}\NormalTok{start) }\SpecialCharTok{/} \DecValTok{60}\NormalTok{)}
\NormalTok{new\_df}\SpecialCharTok{$}\NormalTok{count\_per\_minute }\OtherTok{\textless{}{-}}\NormalTok{ new\_df}\SpecialCharTok{$}\NormalTok{total\_count }\SpecialCharTok{/}\NormalTok{ new\_df}\SpecialCharTok{$}\NormalTok{duration}

\DocumentationTok{\#\#\# add a column to label land use}
\NormalTok{new\_df}\SpecialCharTok{$}\NormalTok{land\_use }\OtherTok{\textless{}{-}} \FunctionTok{ifelse}\NormalTok{(new\_df}\SpecialCharTok{$}\NormalTok{classification }\SpecialCharTok{==} \DecValTok{1}\NormalTok{, }\StringTok{"Mixed Use"}\NormalTok{, }\StringTok{"Single Use"}\NormalTok{)}

\DocumentationTok{\#\#\# remove NA rows}
\NormalTok{new\_df\_clean }\OtherTok{\textless{}{-}} \FunctionTok{na.omit}\NormalTok{(new\_df)}

\DocumentationTok{\#\#\# Install and load the \textquotesingle{}moments\textquotesingle{} package}
\FunctionTok{library}\NormalTok{(moments)}

\DocumentationTok{\#\# Exploratory Data Analysis}
\DocumentationTok{\#\#\# Descriptive Statistics:}
\DocumentationTok{\#\#\# Compute variance, standard deviation, skewness, and kurtosis}
\NormalTok{variance }\OtherTok{\textless{}{-}} \FunctionTok{var}\NormalTok{(new\_df\_clean}\SpecialCharTok{$}\NormalTok{total\_count)}
\NormalTok{standard\_deviation }\OtherTok{\textless{}{-}} \FunctionTok{sd}\NormalTok{(new\_df\_clean}\SpecialCharTok{$}\NormalTok{total\_count)}
\NormalTok{skewness }\OtherTok{\textless{}{-}} \FunctionTok{skewness}\NormalTok{(new\_df\_clean}\SpecialCharTok{$}\NormalTok{total\_count)}
\NormalTok{kurtosis }\OtherTok{\textless{}{-}} \FunctionTok{kurtosis}\NormalTok{(new\_df\_clean}\SpecialCharTok{$}\NormalTok{total\_count)}

\DocumentationTok{\#\#\# Print the computed statistics}
\FunctionTok{print}\NormalTok{(}\FunctionTok{paste}\NormalTok{(}\StringTok{\textquotesingle{}Variance: \textquotesingle{}}\NormalTok{, variance))}
\end{Highlighting}
\end{Shaded}

\begin{verbatim}
## [1] "Variance:  1349.40416411513"
\end{verbatim}

\begin{Shaded}
\begin{Highlighting}[]
\FunctionTok{print}\NormalTok{(}\FunctionTok{paste}\NormalTok{(}\StringTok{\textquotesingle{}Standard deviation: \textquotesingle{}}\NormalTok{, standard\_deviation))}
\end{Highlighting}
\end{Shaded}

\begin{verbatim}
## [1] "Standard deviation:  36.7342369475007"
\end{verbatim}

\begin{Shaded}
\begin{Highlighting}[]
\FunctionTok{print}\NormalTok{(}\FunctionTok{paste}\NormalTok{(}\StringTok{\textquotesingle{}Skewness: \textquotesingle{}}\NormalTok{, skewness))}
\end{Highlighting}
\end{Shaded}

\begin{verbatim}
## [1] "Skewness:  3.01251530071116"
\end{verbatim}

\begin{Shaded}
\begin{Highlighting}[]
\FunctionTok{print}\NormalTok{(}\FunctionTok{paste}\NormalTok{(}\StringTok{\textquotesingle{}Kurtosis: \textquotesingle{}}\NormalTok{, kurtosis))}
\end{Highlighting}
\end{Shaded}

\begin{verbatim}
## [1] "Kurtosis:  16.4943502560228"
\end{verbatim}

\begin{Shaded}
\begin{Highlighting}[]
\DocumentationTok{\#\#\# Correlation matrix among the quantitative variables}
\NormalTok{correlation\_matrix }\OtherTok{\textless{}{-}} \FunctionTok{cor}\NormalTok{(new\_df\_clean[,}\FunctionTok{sapply}\NormalTok{(new\_df\_clean, is.numeric)])}
\FunctionTok{print}\NormalTok{(correlation\_matrix)}
\end{Highlighting}
\end{Shaded}

\begin{verbatim}
##                           id total_count classification     duration
## id                1.00000000  0.28570061    0.216875010 -0.030594322
## total_count       0.28570061  1.00000000    0.033722809 -0.068682289
## classification    0.21687501  0.03372281    1.000000000 -0.003323461
## duration         -0.03059432 -0.06868229   -0.003323461  1.000000000
## count_per_minute         NaN         NaN            NaN          NaN
##                  count_per_minute
## id                            NaN
## total_count                   NaN
## classification                NaN
## duration                      NaN
## count_per_minute                1
\end{verbatim}

\begin{Shaded}
\begin{Highlighting}[]
\DocumentationTok{\#\#\# Boxplot}
\FunctionTok{ggplot}\NormalTok{(new\_df\_clean, }\FunctionTok{aes}\NormalTok{(}\AttributeTok{x =}\NormalTok{ land\_use, }\AttributeTok{y =}\NormalTok{ total\_count, }\AttributeTok{fill =}\NormalTok{ land\_use)) }\SpecialCharTok{+}
  \FunctionTok{geom\_boxplot}\NormalTok{() }\SpecialCharTok{+}
  \FunctionTok{xlab}\NormalTok{(}\StringTok{"Land Use Diversity"}\NormalTok{) }\SpecialCharTok{+}
  \FunctionTok{ylab}\NormalTok{ (}\StringTok{"Number of People Moving"}\NormalTok{) }\SpecialCharTok{+}
  \FunctionTok{theme}\NormalTok{(}\AttributeTok{legend.position =} \StringTok{"none"}\NormalTok{)}
\end{Highlighting}
\end{Shaded}

\includegraphics{state321_final_files/figure-latex/unnamed-chunk-2-1.pdf}

\begin{Shaded}
\begin{Highlighting}[]
\DocumentationTok{\#\#\# Violin plot with different colors for single{-}use and mixed{-}use}
\FunctionTok{ggplot}\NormalTok{(new\_df\_clean, }\FunctionTok{aes}\NormalTok{(}\AttributeTok{x =}\NormalTok{ land\_use, }\AttributeTok{y =}\NormalTok{ total\_count, }\AttributeTok{fill =}\NormalTok{ land\_use)) }\SpecialCharTok{+}
  \FunctionTok{geom\_violin}\NormalTok{() }\SpecialCharTok{+}
  \FunctionTok{geom\_boxplot}\NormalTok{(}\AttributeTok{width=}\FloatTok{0.1}\NormalTok{, }\AttributeTok{fill=}\StringTok{"white"}\NormalTok{) }\SpecialCharTok{+}
  \FunctionTok{xlab}\NormalTok{(}\StringTok{"Land Use Diversity"}\NormalTok{) }\SpecialCharTok{+}
  \FunctionTok{ylab}\NormalTok{ (}\StringTok{"Number of People Moving"}\NormalTok{) }\SpecialCharTok{+}
  \FunctionTok{theme}\NormalTok{(}\AttributeTok{legend.position =} \StringTok{"none"}\NormalTok{)}
\end{Highlighting}
\end{Shaded}

\includegraphics{state321_final_files/figure-latex/unnamed-chunk-2-2.pdf}

\hypertarget{section-three-results}{%
\subsection{Section Three: Results}\label{section-three-results}}

\hypertarget{a-results-section-containing-a-scatterplot-of-the-main-relationship-of-interest-and-output-for-the-main-regression}{%
\subsubsection{A results section containing a scatterplot of the main
relationship of interest and output for the main
regression}\label{a-results-section-containing-a-scatterplot-of-the-main-relationship-of-interest-and-output-for-the-main-regression}}

\hypertarget{statistical-analysis}{%
\subsection{Statistical Analysis}\label{statistical-analysis}}

\hypertarget{bivariate-regression-total_count-classification}{%
\subsubsection{Bivariate regression: total\_count \textasciitilde{}
classification}\label{bivariate-regression-total_count-classification}}

\begin{Shaded}
\begin{Highlighting}[]
\NormalTok{model1 }\OtherTok{\textless{}{-}} \FunctionTok{lm}\NormalTok{(total\_count }\SpecialCharTok{\textasciitilde{}}\NormalTok{ classification, }\AttributeTok{data =}\NormalTok{ new\_df\_clean)}
\FunctionTok{summary}\NormalTok{(model1)  }
\end{Highlighting}
\end{Shaded}

\begin{verbatim}
## 
## Call:
## lm(formula = total_count ~ classification, data = new_df_clean)
## 
## Residuals:
##     Min      1Q  Median      3Q     Max 
## -32.227 -22.227 -11.227   6.773 301.773 
## 
## Coefficients:
##                Estimate Std. Error t value Pr(>|t|)   
## (Intercept)      24.050      8.214   2.928   0.0035 **
## classification    8.177      8.312   0.984   0.3255   
## ---
## Signif. codes:  0 '***' 0.001 '**' 0.01 '*' 0.05 '.' 0.1 ' ' 1
## 
## Residual standard error: 36.73 on 850 degrees of freedom
## Multiple R-squared:  0.001137,   Adjusted R-squared:  -3.79e-05 
## F-statistic: 0.9677 on 1 and 850 DF,  p-value: 0.3255
\end{verbatim}

\begin{Shaded}
\begin{Highlighting}[]
\CommentTok{\# Add controls progressively and present the full specification?}
\CommentTok{\# model2 \textless{}{-} lm(total\_count \textasciitilde{} classification + control1, data = new\_df\_clean)}
\CommentTok{\# summary(model2)}

\CommentTok{\# model3 \textless{}{-} lm(total\_count \textasciitilde{} classification + control1 + control2, data = new\_df\_clean)}
\CommentTok{\# summary(model3)}

\CommentTok{\# Final model with all controls}
\CommentTok{\# model4 \textless{}{-} lm(total\_count \textasciitilde{} classification + control1 + control2 + control3, data = new\_df\_clean)}
\CommentTok{\# summary(model4)}
\end{Highlighting}
\end{Shaded}

To illustrate the main relationship of interest, a scatterplot is
provided below. The scatterplot showcases the distribution of pedestrian
counts relative to land use diversity.

\begin{Shaded}
\begin{Highlighting}[]
\CommentTok{\# remove rows with NAs}
\NormalTok{new\_df\_clean }\OtherTok{\textless{}{-}} \FunctionTok{na.omit}\NormalTok{(new\_df)}

\CommentTok{\# generate the scatterplot}
\FunctionTok{ggplot}\NormalTok{(new\_df\_clean, }\FunctionTok{aes}\NormalTok{(}\AttributeTok{x =}\NormalTok{ land\_use, }\AttributeTok{y =}\NormalTok{ total\_count, }\AttributeTok{color =}\NormalTok{ land\_use)) }\SpecialCharTok{+}
  \FunctionTok{geom\_jitter}\NormalTok{(}\AttributeTok{width =} \FloatTok{0.3}\NormalTok{) }\SpecialCharTok{+}
  \FunctionTok{xlab}\NormalTok{(}\StringTok{"Land Use"}\NormalTok{) }\SpecialCharTok{+}
  \FunctionTok{ylab}\NormalTok{(}\StringTok{"Pedestrians Through the Space"}\NormalTok{)}
\end{Highlighting}
\end{Shaded}

\includegraphics{state321_final_files/figure-latex/unnamed-chunk-4-1.pdf}
The principal regression analysis was performed with the `total\_count'
as the outcome variable and `classification' (land use diversity) as the
explanatory variable. The regression output is as follows:

\begin{Shaded}
\begin{Highlighting}[]
\NormalTok{model }\OtherTok{\textless{}{-}} \FunctionTok{lm}\NormalTok{(total\_count }\SpecialCharTok{\textasciitilde{}}\NormalTok{ classification, }\AttributeTok{data =}\NormalTok{ new\_df)}
\FunctionTok{summary}\NormalTok{(model)}
\end{Highlighting}
\end{Shaded}

\begin{verbatim}
## 
## Call:
## lm(formula = total_count ~ classification, data = new_df)
## 
## Residuals:
##     Min      1Q  Median      3Q     Max 
## -36.379 -25.379 -11.379   9.621 297.621 
## 
## Coefficients:
##                Estimate Std. Error t value Pr(>|t|)   
## (Intercept)      19.531      7.207   2.710  0.00678 **
## classification   16.848      7.256   2.322  0.02032 * 
## ---
## Signif. codes:  0 '***' 0.001 '**' 0.01 '*' 0.05 '.' 0.1 ' ' 1
## 
## Residual standard error: 40.77 on 2376 degrees of freedom
##   (4 observations deleted due to missingness)
## Multiple R-squared:  0.002264,   Adjusted R-squared:  0.001844 
## F-statistic: 5.391 on 1 and 2376 DF,  p-value: 0.02032
\end{verbatim}

The model indicates a significant positive relationship between land use
diversity and pedestrian activity. This relationship suggests that a
higher level of land use diversity corresponds to an increase in
pedestrian movement.

\hypertarget{conclusion}{%
\subsection{Conclusion}\label{conclusion}}

\hypertarget{a-brief-conclusion-summarizing-the-results-assesses-the-extent-to-which-you-found-support-for-your-hypothesis-and-describes-limitations-of-your-analysis-and-threats-to-inference}{%
\subsubsection{A brief conclusion summarizing the results, assesses the
extent to which you found support for your hypothesis, and describes
limitations of your analysis and threats to
inference}\label{a-brief-conclusion-summarizing-the-results-assesses-the-extent-to-which-you-found-support-for-your-hypothesis-and-describes-limitations-of-your-analysis-and-threats-to-inference}}

The analysis provides evidence supporting our hypothesis that land use
diversity positively influences pedestrian activity in Seattle. This
finding has significant implications for urban development strategies
and the promotion of pedestrian-friendly environments.

However, these results should be interpreted with some limitations in
mind. The study is observational and, as such, cannot confirm causality.
Additionally, the model explains a relatively small portion of the
variance in pedestrian activity, indicating that other factors beyond
land use diversity likely contribute significantly to pedestrian
movement.

Moving forward, research may consider incorporating more variables into
the analysis, such as population density, available amenities, and
public transportation facilities, to gain a more comprehensive
understanding of pedestrian activity determinants.

Despite the limitations, this analysis provides valuable insights into
urban design and planning, hinting that land use diversity can foster
more vibrant, pedestrian-friendly environments. The results of this
study provide a stepping stone for future research exploring urban
environments and pedestrian activity.

\end{document}
